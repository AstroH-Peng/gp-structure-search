\newcommand{\fhbig}{2.25cm}
\newcommand{\fwbig}{2.4cm}
\newcommand{\kernpic}[1]{\includegraphics[height=\fhbig,width=\fwbig]{../figures/structure_examples/#1}}
\newcommand{\kernpicr}[1]{\rotatebox{90}{\includegraphics[height=\fwbig,width=\fhbig]{../figures/structure_examples/#1}}}
\newcommand{\largeplus}{\tabbox{{\Large+}}}
\newcommand{\largeeq}{\tabbox{{\Large=}}}
\newcommand{\largetimes}{\tabbox{{\Large$\times$}}}
\begin{figure}
\centering
\renewcommand{\tabularxcolumn}[1]{>{\arraybackslash}m{#1}}
%\begin{tabular}{m{\fwbig}m{0.01\textwidth}m{\fwbig}m{0.01\textwidth}m{\fwbig}m{\fwbig}m{\fwbig}}
\begin{tabular}{m{\fwbig}m{\fwbig}m{\fwbig}}
%\begin{tabularx}{\textwidth}{XXXXX|X|X}
composite \newline kernel & draws from GP & GP posterior \\ \toprule
\kernpicr{se_times_per} & \kernpic{se_times_per_draws} & \kernpic{se_times_per_post} \\
SE $\times$ Per & locally periodic & \\  \midrule
 \kernpicr{se_plus_per} & \kernpic{se_plus_per_draws} & \kernpic{se_plus_per_post} \\
SE + Per & perdiodic with \newline local noise &  \\ \midrule
\kernpic{lin_times_lin} & \kernpic{lin_times_lin_draws} & \kernpic{lin_times_lin_post} \\
Lin $\times$ Lin  & quadratic &  \\ \midrule
 \kernpicr{se_times_lin} & \kernpic{se_times_lin_draws} & \kernpic{se_times_lin_post} \\
%Lin $\times$ SE  & increasing \newline local variation &  \\ \midrule
% \kernpic{lin_times_per} & \kernpic{lin_times_per_draws} & \kernpic{lin_times_per_post} \\
Lin $\times$ Per  & growing \newline amplitude &  \\ \midrule
 \kernpic{lin_plus_per} & \kernpic{lin_plus_per_draws} & \kernpic{lin_plus_per_post} \\
Lin + Per & Per with trend &  \\
\end{tabular}
\caption{ Examples of prior structures that can be expressed by
  composing kernels.  The x-axis has the same scale for all plots.
  The 3rd column shows the composite kernel function, the 4th column
  shows draws from a zero-mean GP with that kernel, and the right-most
  column shows a GP posterior after conditioning on two
  datapoints.
}
\label{fig:kernels}
\end{figure}
