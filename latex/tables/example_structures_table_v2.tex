\newcommand{\fhbig}{1.8cm}
\newcommand{\fwbig}{2.4cm}
\newcommand{\kernpic}[1]{\includegraphics[height=\fhbig,width=\fwbig]{../figures/structure_examples/#1}}
\newcommand{\kernpicr}[1]{\rotatebox{90}{\includegraphics[height=\fwbig,width=\fhbig]{../figures/structure_examples/#1}}}
\newcommand{\largeplus}{\tabbox{{\Large+}}}
\newcommand{\largeeq}{\tabbox{{\Large=}}}
\newcommand{\largetimes}{\tabbox{{\Large$\times$}}}
\begin{figure}
\centering
\renewcommand{\tabularxcolumn}[1]{>{\arraybackslash}m{#1}}
%\begin{tabular}{m{\fwbig}m{0.01\textwidth}m{\fwbig}m{0.01\textwidth}m{\fwbig}m{\fwbig}m{\fwbig}}
\begin{tabular}{ccc}%{m{\fwbig}m{\fwbig}m{\fwbig}}
%\begin{tabularx}{\textwidth}{XXXXX|X|X}
%Composite & Draws from \gp{} & \gp{} posterior \\ \toprule
\kernpicr{se_times_per} & \kernpic{se_times_per_draws} & \kernpic{se_times_per_post} \\
SE $\times$ Per & \multicolumn{2}{c}{locally periodic} \\  \midrule
% \kernpicr{se_plus_per} & \kernpic{se_plus_per_draws} & \kernpic{se_plus_per_post} \\
% SE + Per & perdiodic with \newline local noise &  \\ \midrule
\kernpic{lin_times_lin} & \kernpic{lin_times_lin_draws} & \kernpic{lin_times_lin_post} \\
Lin $\times$ Lin  & \multicolumn{2}{c}{quadratic functions} \\ \midrule
% \kernpicr{se_times_lin} & \kernpic{se_times_lin_draws} & \kernpic{se_times_lin_post} \\
% Lin $\times$ SE  & increasing \newline local variation &  \\ \midrule
 \kernpic{lin_times_per} & \kernpic{lin_times_per_draws} & \kernpic{lin_times_per_post} \\
Lin $\times$ Per  & \multicolumn{2}{c}{growing amplitude}  \\ \midrule
 \kernpic{lin_plus_per} & \kernpic{lin_plus_per_draws} & \kernpic{lin_plus_per_post} \\
Lin + Per & \multicolumn{2}{c}{periodic with trend}  \\
\end{tabular}
\caption{ Examples of structures expressible by
  composite kernels.  The x-axis has the same scale for all plots.
  Left column: the composite kernel  Center: draws from a \gp{} with that kernel.  Right: \gp{} posterior after conditioning on three
  datapoints.
}
\label{fig:kernels}
\end{figure}
