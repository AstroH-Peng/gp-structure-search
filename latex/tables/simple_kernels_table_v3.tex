\newcommand{\fhbig}{1.6cm}
\newcommand{\fwbig}{1.8cm}
\newcommand{\kernpic}[1]{\includegraphics[height=\fhbig,width=\fwbig]{../figures/structure_examples/#1}}
\newcommand{\kernpicr}[1]{\rotatebox{90}{\includegraphics[height=\fwbig,width=\fhbig]{../figures/structure_examples/#1}}}
\newcommand{\addkernpic}[1]{{\includegraphics[height=\fhbig,width=\fwbig]{../figures/additive_multi_d/#1}}}
\newcommand{\largeplus}{\tabbox{{\Large+}}}
\newcommand{\largeeq}{\tabbox{{\Large=}}}
\newcommand{\largetimes}{\tabbox{{\Large$\times$}}}
\begin{figure}[ht]
\centering
\renewcommand{\tabularxcolumn}[1]{>{\arraybackslash}m{#1}}
%\begin{tabular}{m{\fwbig}m{0.01\textwidth}m{\fwbig}m{0.01\textwidth}m{\fwbig}m{\fwbig}m{\fwbig}}
%\begin{tabular}{C{\fwbig}C{\fwbig}C{\fwbig}C{\fwbig}}%{m{\fwbig}m{\fwbig}m{\fwbig}}
\begin{tabularx}{\columnwidth}{XXXX}
%Composite & Draws from \gp{} & \gp{} posterior \\ \toprule
  \kernpic{se_kernel} & \kernpic{se_kernel_draws}
& \kernpic{per_kernel} & \kernpic{per_kernel_draws_s2}
\\
  {\small Squared-exp (\kSE)} & {\small local variation} 
& {\small Periodic (\kPer)} & {\small repeating structure}
\\
\midrule
  \kernpic{lin_kernel} & \kernpic{lin_kernel_draws}
& \kernpic{rq_kernel} & \kernpic{rq_kernel_draws}
\\
  {\small Linear (\kLin)} & {\small linear functions} 
& {\small Rational-quadratic(\kRQ)} & {\small multi-scale variation}
\end{tabularx}
\caption{ 
%Examples of structures expressible by composite kernels.  
%The x-axis has the same scale for all plots.
  Left column and third columns: base kernels $k(\cdot,0)$.  Second and fourth columns: draws from a \gp{} with each repective kernel.  The x-axis has the same range on all plots.}
\label{fig:basic_kernels}
\end{figure}

