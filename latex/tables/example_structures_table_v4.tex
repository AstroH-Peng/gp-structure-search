%\newcommand{\fhbig}{1.6cm}
%\newcommand{\fwbig}{1.8cm}
%\newcommand{\kernpic}[1]{\includegraphics[height=\fhbig,width=\fwbig]{../figures/structure_examples/#1}}
%\newcommand{\kernpicr}[1]{\rotatebox{90}{\includegraphics[height=\fwbig,width=\fhbig]{../figures/structure_examples/#1}}}
%\newcommand{\addkernpic}[1]{{\includegraphics[height=\fhbig,width=\fwbig]{../figures/additive_multi_d/#1}}}
%\newcommand{\largeplus}{\tabbox{{\Large+}}}
%\newcommand{\largeeq}{\tabbox{{\Large=}}}
%\newcommand{\largetimes}{\tabbox{{\Large$\times$}}}
\begin{figure}[ht]
\centering
\renewcommand{\tabularxcolumn}[1]{>{\arraybackslash}m{#1}}
%\begin{tabular}{m{\fwbig}m{0.01\textwidth}m{\fwbig}m{0.01\textwidth}m{\fwbig}m{\fwbig}m{\fwbig}}
%\begin{tabular}{C{\fwbig}C{\fwbig}C{\fwbig}C{\fwbig}}%{m{\fwbig}m{\fwbig}m{\fwbig}}
\begin{tabularx}{\columnwidth}{XXXX}
  \kernpic{se_times_per} & \kernpic{se_times_per_draws_s1}
& \kernpic{lin_times_lin} & \kernpic{lin_times_lin_draws} 
\\
  {\small $\kSE \times \kPer$} & {\small locally \newline periodic} 
& {\small $\kLin \times \kLin$} & {\small quadratic functions}
\\
\midrule
  \kernpic{se_plus_per} & \kernpic{se_plus_per_draws_s3}
& \kernpic{lin_plus_per} & \kernpic{lin_plus_per_draws}
\\
  {\small $\kSE + \kPer$ } & {\small periodic with noise}
& {\small $\kLin + \kPer$} & {\small periodic with trend}
\\
\midrule
  \kernpic{se_times_lin} & \kernpic{se_times_lin_draws_s2}
& \kernpic{lin_times_per} & \kernpic{lin_times_per_draws_s2}
\\
  {\small $\kLin \times \kSE$} & {\small increasing variation}
& {\small $\kLin \times \kPer$} & {\small growing amplitude}
\\
\midrule
  \addkernpic{additive_kernel} & \addkernpic{additive_kernel_draw_sum}
& \addkernpic{sqexp_kernel}  & \addkernpic{sqexp_draw}
\\
  {\small $\kSE_1 + \kSE_2$} & {\small $f_1(x_1)$ $+ f_2(x_2)$}
& {\small $\kSE_1 \times \kSE_2$} & {\small $f(x_1, x_2)$}
\end{tabularx}
\caption{ Examples of structures expressible by
  composite kernels.  
%The x-axis has the same scale for all plots.
  Left column and third columns: composite kernels $k(\cdot,0)$.  Plots have same meaning as in Figure \ref{fig:basic_kernels}.}
\label{fig:kernels}
\end{figure}


